\documentclass[a4paper,10pt,twocolumn,oneside]{article}
\setlength{\columnsep}{10pt}                                                                    %兩欄模式的間距
\setlength{\columnseprule}{0pt}                                                                %兩欄模式間格線粗細

\usepackage{amsthm}								%定義,例題
\usepackage{amssymb}
%\usepackage[margin=2cm]{geometry}
\usepackage{fontspec}								%設定字體
\usepackage{color}
\usepackage[x11names]{xcolor}
\usepackage{listings}								%顯示code用的
%\usepackage[Glenn]{fncychap}						%排版,頁面模板
\usepackage{fancyhdr}								%設定頁首頁尾
\usepackage{graphicx}								%Graphic
\usepackage{enumerate}
\usepackage{titlesec}
\usepackage{amsmath}
\usepackage[CheckSingle, CJKmath]{xeCJK}
% \usepackage{CJKulem}
\usepackage[titles]{tocloft}

%\usepackage[T1]{fontenc}
\usepackage{amsmath, courier, listings, fancyhdr, graphicx}
\topmargin=0pt
\headsep=5pt
\textheight=780pt
\footskip=0pt
\voffset=-40pt
\textwidth=545pt
\marginparsep=0pt
\marginparwidth=0pt
\marginparpush=0pt
\oddsidemargin=0pt
\evensidemargin=0pt
\hoffset=-42pt

\titlespacing\subsection{0pt}{5pt plus 4pt minus 2pt}{0pt plus 2pt minus 2pt}


%\renewcommand\listfigurename{圖目錄}
%\renewcommand\listtablename{表目錄} 

%%%%%%%%%%%%%%%%%%%%%%%%%%%%%

\setmainfont{Victor Mono}				%主要字型
%\setmonofont{Monaco}				%主要字型
\setmonofont{Victor Mono}
\setCJKmainfont{Noto Sans CJK TC}
% \setCJKmainfont{Consolas}			%中文字型
%\setmainfont{sourcecodepro}
\XeTeXlinebreaklocale "zh"						%中文自動換行
\XeTeXlinebreakskip = 0pt plus 1pt				%設定段落之間的距離
\setcounter{secnumdepth}{3}						%目錄顯示第三層
\setlength\cftsecnumwidth{1em}

%%%%%%%%%%%%%%%%%%%%%%%%%%%%%
\makeatletter
\lst@CCPutMacro\lst@ProcessOther {"2D}{\lst@ttfamily{-{}}{-{}}}
\@empty\z@\@empty
\makeatother
\lstset{											% Code顯示
language=C++,										% the language of the code
basicstyle=\footnotesize\ttfamily, 						% the size of the fonts that are used for the code
%numbers=left,										% where to put the line-numbers
numberstyle=\footnotesize,						% the size of the fonts that are used for the line-numbers
stepnumber=1,										% the step between two line-numbers. If it's 1, each line  will be numbered
numbersep=5pt,										% how far the line-numbers are from the code
backgroundcolor=\color{white},					% choose the background color. You must add \usepackage{color}
showspaces=false,									% show spaces adding particular underscores
showstringspaces=false,							% underline spaces within strings
showtabs=false,									% show tabs within strings adding particular underscores
frame=false,											% adds a frame around the code
tabsize=2,											% sets default tabsize to 2 spaces
captionpos=b,										% sets the caption-position to bottom
breaklines=true,									% sets automatic line breaking
breakatwhitespace=false,							% sets if automatic breaks should only happen at whitespace
escapeinside={\%*}{*)},							% if you want to add a comment within your code
morekeywords={constexpr},									% if you want to add more keywords to the set
keywordstyle=\bfseries\color{Blue1},
commentstyle=\itshape\color{Red4},
stringstyle=\itshape\color{Green4},
}

%%%%%%%%%%%%%%%%%%%%%%%%%%%%%

\begin{document}
\pagestyle{fancy}
\fancyfoot{}
%\fancyfoot[R]{\includegraphics[width=20pt]{ironwood.jpg}}
\fancyhead[L]{PixelCat's codebook / stole from NTU 8BQube}
\fancyhead[R]{\thepage}
\renewcommand{\headrulewidth}{0.4pt}
\renewcommand{\contentsname}{ToC} 

\scriptsize
\tableofcontents
%%%%%%%%%%%%%%%%%%%%%%%%%%%%%

%\newpage

% format all .cpp code:
% $ clang-format -style=file -i **/*.cpp

\section{General}
\subsection{vimrc}
\lstinputlisting[language={}]{1-general/vimrc}
\subsection{Default Code}
\lstinputlisting{1-general/default.cpp}
\subsection{Splitmix64}
\lstinputlisting{1-general/splitmix64.cpp}
\subsection{Optimization}
\lstinputlisting{1-general/optimize.cpp}

\section{Math \& Polynomials}
\subsection{Theorems}
\begin{itemize}

    \item Pick's Theorem
    
    if a polygon has integer coords for all vertices, then
    $$ A = I + \frac{B}{2} - 1 $$
    , where $A$ = area, $I$ = internal points, $B$ = points on boundary.

\end{itemize}

\subsection{Fast pow}
\lstinputlisting{2-math/fast_pow.cpp}
\subsection{Extended GCD}
\lstinputlisting{2-math/exgcd.cpp}
\subsection{Prime Sieve}
\lstinputlisting{2-math/prime_sieve.cpp}
\subsection{Prime Numbers}
\lstinputlisting{2-math/primes.txt}
\subsection{Mod Inverse}
\lstinputlisting{2-math/mod_inverse.cpp}
\subsection{Chinese Remainder Theorem}
\lstinputlisting{2-math/crt.cpp}
% tested on Library Checker convolution_mod_1000000007
\subsection{Millar Robin}
\lstinputlisting{2-math/millar_robin.cpp}
% tested on Library Checker factorize
\subsection{Pollard's Rho}
\lstinputlisting{2-math/pollards_rho.cpp}
% tested on Library Checker factorize
\subsection{Polynomials}
\subsubsection{FFT*}
\lstinputlisting{2-math/fft.cpp}
% not tested!
\subsubsection{NTT}
\lstinputlisting{2-math/ntt.cpp}
% tested on Library Checker convolution_mod_1000000007 and CSES 2111/2112/2113

\section{Geometry}
\subsection{Vector Operations}
\lstinputlisting{3-geometry/vector.cpp}
\subsection{Convex Hull}
\lstinputlisting{3-geometry/convex_hull.cpp}
% tested on CSES 2195
\subsection{Polar Angle Sorting}
\lstinputlisting{3-geometry/polar_sort.cpp}

\section{Data Structure}
\subsection{BIT}
\lstinputlisting{4-ds/bit.cpp}
\subsection{Segment Tree with Lazy Tags}
\lstinputlisting{4-ds/lazy_segtree.cpp}
% tested on Library Checker range_affine_range_sum
\subsection{Treap}
\lstinputlisting{4-ds/treap.cpp}
% tested on CSES 2074

\section{Graph \& Flow}
\subsection{Edge BCC}
\lstinputlisting{5-graph/tarjan_edge_bcc.cpp}
% tested on Library Checker two_edge_connected_components and CSES 2076
\subsection{Articulation Points}
\lstinputlisting{5-graph/tarjan_vertex_bcc.cpp}
% tested on CSES 2077
\subsection{SCC}
\lstinputlisting{5-graph/scc.cpp}
% tested on Library Checker scc and CSES 1684
\subsection{Dinic}
\lstinputlisting{5-graph/dinic.cpp}
% tested on CSES 1694/1695/1696/1711
\subsection{Min Cost Max Flow}
\lstinputlisting{5-graph/mcmf.cpp}

\section{String}
\subsection{Z Algorithm}
\lstinputlisting{6-string/z.cpp}
% tested on Library Checker zalgorithm and CSES 1753
\subsection{KMP}
\lstinputlisting{6-string/kmp.cpp}
% tested on CSES 1753
\subsection{Manacher's Algorithm}
\lstinputlisting{6-string/manacher.cpp}
% tested on Library Checker enumerate_palindromes
\subsection{Minimal Rotation}
\lstinputlisting{6-string/minimal_rotation.cpp}
% tested on CSES 1110
\subsection{SA \& LCP}
\lstinputlisting{6-string/sa_lcp.cpp}
% tested on Library Checker suffixarray/number_of_substrings



\end{document}
